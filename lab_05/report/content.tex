% Содержимое отчета по курсу Анализ алгоритмов

\aaunnumberedsection{ВВЕДЕНИЕ}{sec:intro}

Параллельные вычисления – это использование нескольких или многих вычислительных устройств для одновременного выполнения разных частей одной программы~\cite{gafarov_paral}.

Цель работы --- получение навыка организации параллельных вычислений по конвейерному принципу.

Задачи работы:
\begin{itemize}
	\item[---] определение структуры html файла, содержащего рецепт;
	\item[---] разработка программного обеспечения, выполняющего обработку рецептов в конвейерном режиме;
	\item[---] анализ времен ожидания в очередях и обработки на каждом этапе.
\end{itemize}


\aasection{Входные и выходные данные}{sec:input-output}

Входными данными являются html файлы с рецептами, загруженные с сайта https://www.gotovim.ru. Каждый файл содержит ровно один рецепт. Выходными данными являются записи характеристик каждой из выполненных задач в базе данных.


\aasection{Преобразование входных данных в выходные}{sec:algorithm}

Программа читает данные из файла, извлекает необходимые, записывает извлеченные данные в базу данных. Реализация алгоритма извлечения данных представлена в листинге~\ref{lst:parse.cpp}.

\includelisting
{parse.cpp} % Имя файла с расширением (файл должен быть расположен в директории inc/lst/)
{Функции разбирают входные данные и извлекают записи} % Подпись листинга

\aasection{Примеры работы программы}{sec:demo}
На листинге~\ref{lst:exemple.txt} представлен пример работы программы. На вход программа получает три html файла. Каждый файл проходит следующие этапы обработки: чтение, извлечение данных, запись в базу данных.

\includelisting
{exemple.txt} % Имя файла с расширением (файл должен быть расположен в директории inc/lst/)
{Примеры работы программы} % Подпись листинга


\aasection{Тестирование}{sec:tests}

Тестирование программы проводилось на 100 входных html файлов с различными рецептами. Проверка 5 случайных файлов: сравнивались данные, извлеченные программой, с эталонными значениями в html файлах. Тестирование было успешно пройдено.

\aasection{Описание исследования}{sec:study}

Технические характеристики устройства:

\begin{itemize}
	\item[---] операционная система Manjaro Linux x86\_64;
	\item[---] процессор Ryzen 5500U 6 ядер, тактовая частота 2.1 ГГц;
	\item[---] оперативная память 16 Гбайт.
\end{itemize}

В ходе исследования были получены следующие характеристики:

\begin{itemize}
	\item[---] среднее время существования задачи;
	\item[---] среднее время ожидания задачи в каждой из очередей;
	\item[---] среднее время обработки задачи на каждой из стадий.
\end{itemize}

Исследование проводилось для 100 html файлов.

\FloatBarrier
\begin{table}[h!]
	\centering
	\caption{Среднее значение времен на каждом этапе}
	\begin{tabular}{|l|c|}
		\hline Этап выполнения & Среднее время, мс \\\hline
		Ожидание в очереди чтения & 70.35 \\\hline
		Ожидание в очереди извлечения & 1412.39 \\\hline
		Ожидание в очереди записи & 0.00 \\\hline
		Обработка чтения & 2.00 \\\hline
		Обработка извлечения & 29.42 \\\hline
		Обработка записи & 1.04 \\\hline
		Время жизни задачи & 1516.97 \\\hline
	\end{tabular}
	\label{tab:time_spent}
\end{table}
\FloatBarrier

По результатам проведенного исследования сделан вывод о том, что при организации параллельных вычислений по конвейерному принципу необходимо уделять особое внимание "узким" местам, из-за которых возникают простои остальных частей конвейера в ожидании поступления новых данных. 

В данной лабораторной работе извлечение данных вызывает простой, и для улучшения скорости обработки необходимо модифицировать именно эту часть программы.

\aaunnumberedsection{ЗАКЛЮЧЕНИЕ}{sec:outro}

Получены навыки организации параллельных вычислений по конвейерному принципу. Цель работы достигнута. Решены все поставленные задачи: 
\begin{itemize}
	\item[---] определена структура html файла, содержащего рецепт;
	\item[---] разработано программное обеспечение, выполняющее обработку рецептов в конвейерном режиме;
	\item[---] проведен анализ времен ожидания в очередях и обработки на каждом этапе.
\end{itemize}
