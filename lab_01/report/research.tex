\chapter{Исследовательская часть}

В данной части будут использоваться следующие обозначения для различных алгоритмов поиска расстояний.

\begin{itemize}
 	\item[---]А1 --- матричный алгоритм поиска расстояния Левенштейна
 	\item[---]А2 --- рекурсивный алгоритм поиска расстояния Левенштейна
 	\item[---]А3 --- матричный алгоритм поиска расстояния Дамерау~--~Левенштейна
\end{itemize}


\section{Технические характеристики}

Технические характеристики устройства:

\begin{itemize}
	\item[---] операционная система Manjaro Linux x86\_64;
	\item[---] процессор Ryzen 5500U 6 ядер, тактовая частота 2.1 ГГц;
	\item[---] оперативная память 16 Гбайт.
\end{itemize}

При тестировании ноутбук был включён в сеть электропитания. Во время тестирования ноутбук был нагружен только системными приложениями окружения, а также системой тестирования.


\section{Сравнительный анализ временных затрат}

Замеры времени для каждой длины слов проводились 100 раз. Результат замера --- среднее арифметическое время работы алгоритма, на вход подавались сгенерированные случайным образом строки.

В таблице~\ref{tbl:time} представлены результаты измерений времени работы алгоритмов в зависимости от длины исходных строк. 

\clearpage

\begin{center}
    \captionsetup{justification=raggedright,singlelinecheck=off}
    \begin{longtable}[c]{|c|c|c|c|}
    \caption{Временные затраты\label{tbl:time}} \\ \hline
		Длина & A1 & A2 & A3 \\ \hline
		1 & 0.0035 & 0.0020 & 0.0034 \\\hline
		2 & 0.0018 & 0.0013 & 0.0017 \\\hline
		3 & 0.0028 & 0.0036 & 0.0029 \\\hline
		4 & 0.0045 & 0.0147 & 0.0048 \\\hline
		5 & 0.0064 & 0.0728 & 0.0073 \\\hline
		6 & 0.0091 & 0.3495 & 0.0101 \\\hline
		7 & 0.0122 & 1.6686 & 0.0143 \\\hline
		8 & 0.0159 & 9.0077 & 0.0182 \\\hline
		9 & 0.0201 & 44.0207 & 0.0233 \\\hline
		10 & 0.0249 & 248.8653 & 0.0290 \\\hline
	\end{longtable}
\end{center}

На рисунке~\ref{img:graph_time} представлено сравнение временных затрат для алгоритмов матричного поиска расстояния Левенштейна, рекурсивного поиска расстояния Левенштейна и матричного поиска расстояния Дамерау~--~Левенштейна.

\imgHeight{90mm}{graph_time}{Временные затраты}


\section{Сравнительный анализ затрат памяти}
Для проведения замеров объёма затрачиваемаемой памяти, на вход подавались сгенерированные случайным образом строки различной длины.

В таблице~\ref{tbl:memory} представлены результаты измерений затрат памяти при работе алгоритмов в зависимости от длины исходных строк. 

\begin{center}
    \captionsetup{justification=raggedright,singlelinecheck=off}
    \begin{longtable}[c]{|c|c|c|c|}
    \caption{Затраты памяти\label{tbl:memory}} \\ \hline
		Длина & A1 & A2 & A3 \\ \hline
		1 & 128 & 24 & 128 \\\hline
		2 & 192 & 47 & 192 \\\hline
		3 & 232 & 47 & 232 \\\hline
		4 & 288 & 134 & 288 \\\hline
		5 & 392 & 222 & 392 \\\hline
		6 & 480 & 312 & 480 \\\hline
		7 & 584 & 404 & 584 \\\hline
		8 & 704 & 498 & 704 \\\hline
		9 & 904 & 594 & 904 \\\hline
	   10 & 1056 & 692 & 1056 \\\hline
	\end{longtable}
\end{center}

На рисунке~\ref{img:graph_memory} представлено сравнение затрат памяти для алгоритмов матричного поиска расстояния Левенштейна, рекурсивного поиска расстояния Левенштейна и матричного поиска расстояния Дамерау~--~Левенштейна.

\FloatBarrier
\imgHeight{90mm}{graph_memory}{Затраты памяти}
\FloatBarrier


\section{Результаты проведенных исследований}

По полученным данным измерений временных затрат был сделан вывод о том, что рекурсивный алгоритм поиска расстояния Левенштейна имеет экспоненциальный рост временных затрат при увеличении длины строки, что является значительно менее эффективным по сравнению с матричными алгоритмами поиска расстояний Левенштейна и Дамерау~--~Левенштейна.

По полученным данным измерений затрат памяти был сделан вывод о том, что рекурсивный алгоритм поиска расстояния Левенштейна требует меньшего количества затрат памяти, по сравнению с матричными алгоритмами поиска расстояний Левенштейна и Дамерау~--~Левенштейна, которые проигрывают из-за необходимости использовать дополнительную память под матрицу промежуточных значений.

\clearpage
