\chapter{Аналитическая часть}

В данной части работы будут представлены алгоритмы нахождения расстояний Левенштейна и Дамерау~--~Левенштейна.

\section{Расстояние Левенштейна}

Расстояние Левенштейна (англ. Levenshtein distance) (также редакционное расстояние или дистанция редактирования) между двумя строками в теории информации и компьютерной лингвистике --- это минимальное количество операций вставки одного символа, удаления одного символа и замены одного символа на другой, необходимых для превращения одной строки в другую~\cite{lev}.

$\lambda$ --- пустой символ, не входящий ни в одну из рассматриваемых строк.

Обозначения операций:
\begin{itemize}
	\item[---] $w(a, b)$ -- цена замены символа $a$ на символ $b$;
	\item[---] $w(\lambda, b)$ -- цена вставки символа $b$;
	\item[---] $w(a, \lambda)$ -- цена удаления символа $a$.
\end{itemize}

Каждая операция имеет свою цену:
\begin{itemize}
	\item[---] $w(a, a) = 0$;
	\item[---] $w(a, b) = 1$ при $a \neq b$;
	\item[---] $w(\lambda, b) = 1$;
	\item[---] $w(a, \lambda) = 1$.
\end{itemize}

\section{Расстояние Дамерау~--~Левенштейна}

Расстояние Дамерау~--~Левенштейна (англ. Damerau~--~Levenshtein distance) между двумя строками, состоящими из конечного числа символов --- это минимальное число операций вставки, удаления, замены одного символа и транспозиции двух соседних символов, необходимых для перевода одной строки в другую~\cite{damerau_levenstein}.

В алгоритм поиска расстояния Дамерау~--~Левенштейна добавляется ещё одна операция --- транспозиция $w(S_{1}[i], S_{2}[j]) = 1$ при $S_{1}[i] = S_{2}[j + 1]$ и $S_{1}[i + 1] = S_{2}[j]$.


