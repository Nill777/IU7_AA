\usepackage{cmap} % Улучшенный поиск русских слов в полученном pdf-файле
\usepackage[T2A]{fontenc} % Поддержка русских букв
\usepackage[utf8]{inputenc} % Кодировка utf8
\usepackage[english,russian]{babel} % Языки: английский, русский
\usepackage{enumitem} % Оформление списков
\usepackage{placeins} % Оформление графиков
\usepackage{threeparttable} % Создание таблиц

\usepackage[14pt]{extsizes} % 14 шрифт

\usepackage{caption} % Подписи к иллюстрациям и таблицам
\captionsetup{labelsep=endash}
\captionsetup[figure]{name={Рисунок}}

\usepackage{amsmath} % Расширенные математические формулы

\usepackage{geometry} % Отступы
\geometry{left=30mm}
\geometry{right=10mm}
\geometry{top=20mm}
\geometry{bottom=20mm}

\usepackage{titlesec} % Заголовки разделов, глав
\titleformat{\section}
	{\normalsize\bfseries}
	{\thesection}
	{1em}{}
\titlespacing*{\chapter}{0pt}{-30pt}{8pt}
\titlespacing*{\section}{\parindent}{*4}{*4}
\titlespacing*{\subsection}{\parindent}{*4}{*4}

\usepackage{setspace}
\onehalfspacing % Полуторный интервал

\frenchspacing
\usepackage{indentfirst} % Красная строка

\usepackage{titlesec} % Оформление заголовков
\titleformat{\chapter}{\large\bfseries}{\thechapter}{14pt}{\large\bfseries}
\titleformat{\section}{\large\bfseries}{\thesection}{14pt}{\large\bfseries}

\usepackage{listings} % Вставка листингов кода
\usepackage{xcolor}

% Для листинга кода:
\lstset{%
	language=python,   					% выбор языка для подсветки	
	basicstyle=\small\sffamily,			% размер и начертание шрифта для подсветки кода
	numbers=left,						% где поставить нумерацию строк (слева\справа)
	%numberstyle=,						% размер шрифта для номеров строк
	stepnumber=1,						% размер шага между двумя номерами строк
	numbersep=5pt,						% как далеко отстоят номера строк от подсвечиваемого кода
	frame=single,						% рисовать рамку вокруг кода
	tabsize=4,							% размер табуляции по умолчанию равен 4 пробелам
	captionpos=t,						% позиция заголовка вверху [t] или внизу [b]
	breaklines=true,					
	breakatwhitespace=true,				% переносить строки только если есть пробел
	escapeinside={\#*}{*)},				% если нужно добавить комментарии в коде
	backgroundcolor=\color{white},
}

\usepackage{pgfplots} % Инструменты для создания графиков и диаграмм
\usetikzlibrary{datavisualization}
\usetikzlibrary{datavisualization.formats.functions}

\usepackage{graphicx} % Вставка изображений в документ
\newcommand{\imgScale}[3] {
	\begin{figure}[h!]
		\center{\includegraphics[scale=#1]{img/#2}}
		\caption{#3}
		\label{img:#2}
	\end{figure}
}

\usepackage{graphicx}
\newcommand{\imgHeight}[3] {
	\begin{figure}[h!]
		\center{\includegraphics[height=#1]{img/#2}}
		\caption{#3}
		\label{img:#2}
	\end{figure}
}

\usepackage[justification=centering]{caption} % Настройка подписей float объектов

\usepackage[unicode,pdftex]{hyperref} % Ссылки в pdf
\hypersetup{hidelinks}

\usepackage{csvsimple} % Работа с CSV-файлами

\newcommand{\code}[1]{\texttt{#1}} % Определяет новую команду для форматирования фрагментов кода

% Настройки оформления таблиц
\usepackage{longtable}

\usepackage{array}
\usepackage{booktabs}
\usepackage{floatrow}

\floatsetup[longtable]{LTcapwidth=table}


\usepackage{csvsimple}
\usepackage{svg}
