\chapter{Технологическая часть}

\section{Средства реализации}

В данной работе для реализации был выбран язык программирования $Python$. Требуется измерить затрачиваемое время и объём необходимой памяти, для проведения замеров времени использовалась библиотека $time$, объёма затрачиваемаемой памяти использовалась библиотека $tracemalloc$. Для построения графиков использовалась библиотека $mathplotlib$.

\section{Реализация алгоритмов}

В листингах~\ref{lst:lev_matrix},~\ref{lst:lev_rec} и~\ref{lst:dam_lev_matrix} представлены алгоритмы матричного поиска расстояния Левенштейна, рекурсивного поиска расстояния Левенштейна и матричного поиска расстояния Дамерау~--~Левенштейна.
\begin{center}
    \captionsetup{justification=raggedright,singlelinecheck=off}
    \begin{lstlisting}[label=lst:lev_matrix,caption=Матричный алгоритм поиска расстояния Левенштейна]
def levenstein_matrix(str1: str, str2: str) -> int:
	rows = len(str1) + 1
	cols = len(str2) + 1
	matrix = [[0] * cols for i in range(rows)]
	for i in range(1, cols):
		matrix[0][i] = i
	for i in range(1, rows):
		matrix[i][0] = i

	for i in range(1, rows):
		for j in range(1, cols):
			if str1[i - 1] == str2[j - 1]: cost = 0
			else: cost = 1
			matrix[i][j] = min(
				matrix[i][j - 1] + 1, 
				matrix[i - 1][j] + 1, 
				matrix[i - 1][j - 1]  + cost
			)

	return matrix[-1][-1]
\end{lstlisting}
\end{center}

\begin{center}
    \captionsetup{justification=raggedright,singlelinecheck=off}
    \begin{lstlisting}[label=lst:lev_rec,caption=Рекурсивный алгоритм поиска расстояния Левенштейна]
def levenstein_recursive(str1: str, str2: str) -> int:
	if (len(str1) == 0 or len(str2) == 0):
		return max(len(str1), len(str2))
	if str1[0] == str2[0]:
		return levenstein_recursive(str1[1:], str2[1:])
	return 1 + min(
		levenstein_recursive(str1[1:], str2),
		levenstein_recursive(str1[1:], str2[1:]),
		levenstein_recursive(str1, str2[1:])
	)
\end{lstlisting}
\end{center}

\begin{center}
    \captionsetup{justification=raggedright,singlelinecheck=off}
    \begin{lstlisting}[label=lst:dam_lev_matrix,caption=Матричный алгоритм поиска расстояния Дамерау~--~Левенштейна]
def damerau_levenstein_matrix(str1: str, str2: str) -> int:
	rows = len(str1) + 1
	cols = len(str2) + 1
	matrix = [[0] * cols for i in range(rows)]
	for i in range(1, cols):
		matrix[0][i] = i
	for i in range(1, rows):
		matrix[i][0] = i

	for i in range(1, rows):
		for j in range(1, cols):
			if i >= 2 and j >= 2 and str1[i - 1] == str2[j - 2] and str1[i - 2] == str2[j - 1]:
				matrix[i][j] = min(
					matrix[i - 1][j] + 1,
					matrix[i - 1][j - 1] + (0 if str1[i - 1] == str2[j - 1] else 1),
					matrix[i - 2][j - 2] + 1,
					matrix[i][j - 1] + 1,
				)
			else:
				matrix[i][j] = min(
					matrix[i][j - 1] + 1, 
					matrix[i - 1][j] + 1, 
					matrix[i - 1][j - 1]  + (0 if str1[i - 1] == str2[j - 1] else 1)
				)

	return matrix[-1][-1]
\end{lstlisting}
\end{center}


\section{Функциональные тесты}

В таблице~\ref{tbl:functional_test} приведены тесты для тесты для алгоритмов вычисления расстояния. Тесты пройдены успешно.

\begin{center}
    \captionsetup{justification=raggedright,singlelinecheck=off}
    \begin{longtable}[c]{|c|c|p{3cm}|p{3cm}|}
    \caption{Функциональные тесты\label{tbl:functional_test}} \\ \hline
		Строка 1 & Строка 2 & Расстояние Левенштейна & Расстояние Дамерау~--~Левенштейна \\ \hline
		строка & строка & 0 & 0 \\ \hline
		qwerty & ytrewq & 6 & 5 \\ \hline
		звезда & звезды & 1 & 1 \\ \hline
		бобака & бобик & 2 & 2 \\ \hline
		цска & мгту & 4 & 4 \\ \hline
		ерер & рере & 2 & 2 \\ \hline
	\end{longtable}
\end{center}
