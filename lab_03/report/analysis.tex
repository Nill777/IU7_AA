\chapter{Аналитическая часть}
\section{Словарь}

Словарь -- тип данных, который позволяет хранить пары вида «ключ-значение» -- \textit{(k, v)}. 
В паре \textit{(k, v)} -- \textit{v} это значение, которое ассоциируется с ключом \textit{k}.

В данной лабораторной работе словарь представлен следующим образом:
\begin{itemize}
	\item[---] ключ -- VIN номер автомобиля;
	\item[---] значение -- информация об автомобиле.
\end{itemize}


\section{Линейный поиск}

Линейный поиск~\cite{search} -- это метод поиска элемента в списке. Он последовательно проверяет каждый элемент массива до тех пор, пока не будет найдено совпадение или пока не будет просмотрен весь массив.
В массиве из $n$ элементов существует $n + 1$ возможных случаев размещения искомого значения.

Алгоритм при начале работы затрачивает $k$ операций, при сравнении $q$ операций, тогда:
\begin{itemize}
	\item[---] лучший случай -- элемент найден на первом сравнении ($k + q$ операций);
	\item[---] общий случай -- элемент найден на \textit{i-ом} сравнении ($k + i \cdot q$ операций);
	\item[---] худший случай -- элемент найден на последнем сравнении, либо не найден ($k +  n \cdot q$ операций).
\end{itemize}


\section{Бинарный поиск}

Бинарный поиск~\cite{search} -- поиск в заранее отсортированном словаре, который заключается в сравнении со средним ключом, в результате этого сравнения определить, в какой половине словаря находится искомый ключ, и снова применить ту же процедуру к половине словаря.

Алгоритм при начале работы затрачивает $k$ операций, тогда:
\begin{itemize}
	\item[---] лучший случай -- элемент найден на первом сравнении со средним элементом ($b + \log_2 1$);
	\item[---] общий случай -- элемент найден на \textit{i-ом} сравнении ($b + \log_2 i$);
	\item[---] худший случай -- элемент найден на последнем сравнении ($b +  \log_2 n$), где $n$ -- размер словаря.
\end{itemize}
