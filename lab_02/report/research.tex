\chapter{Исследовательская часть}

\section{Технические характеристики}

Технические характеристики устройства:

\begin{itemize}
	\item[---] операционная система Manjaro Linux x86\_64;
	\item[---] процессор Ryzen 5500U 6 ядер, тактовая частота 2.1 ГГц;
	\item[---] оперативная память 16 Гбайт.
\end{itemize}

При тестировании ноутбук был включён в сеть электропитания. Во время тестирования ноутбук был нагружен только системными приложениями окружения, а также системой тестирования.

\section{Сравнительный анализ временных затрат}

Замеры времени для каждой размерности матрицы проводились 100 раз. Результат замера --- среднее арифметическое время работы алгоритма, на вход подавались сгенерированные случайным образом матрицы.

На рисунках~\ref{img:default},~\ref{img:even} и~\ref{img:odd} представлено сравнение временных затрат для стандартного алгоритма умножения матриц, алгоритма Винограда, оптимизированного алгоритма Винограда (общий случай, размерность чётная, размерность нечётная).

\FloatBarrier
\imgHeight{90mm}{default}{Временные затраты, общий случай}
\imgHeight{90mm}{even}{Временные затраты, размерность чётная}
\imgHeight{90mm}{odd}{Временные затраты, размерность нечётная}
\FloatBarrier

\section{Результаты проведенных исследований}

По полученным данным измерений временных затрат был сделан вывод о том, что алгоритм Винограда и оптимизированный алгоритм Винограда, затрачивают меньше времени для обработки по сравнению со стандартным алгоритмом умножения матриц. При увеличении размерности матриц это разрыв становится существеннее.

\clearpage
