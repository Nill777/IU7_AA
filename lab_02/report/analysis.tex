\chapter{Аналитическая часть}
\section{Матрица}

Матрицей $A$ размера $m \times n$ называется прямоугольная таблица чисел, функций или алгебраических выражений, содержащая $m$ строк и $n$ столбцов. Числа $m$ и $n$ определяют размер матрицы~\cite{Belousov2006}.

Умножение матрицы $A$ на матрицу $B$ определено, лишь когда число столбцов первой матрицы в произведении равно числу строк второй. Тогда произведением матриц $\underset{m \times k}{A} \underset{k \times n}{B}$ называется матрица $\underset{m \times n}{C}$, каждый элемент которой $c_{ij}$ равен сумме попарных произведений элементов $i$–й строки матрицы $A$ на соответствующие элементы $j$–го столбца матрицы $B$~\cite{Belousov2006}.

\section{Стандартный алгоритм умножения матриц}

Стандартный алгоритм умножения матриц является одним из базовых методов, используемых для вычисления произведения двух матриц. Пусть даны две матрицы $A$ размером $m \times k$ и $B$ размером $k \times n$. Результатом их умножения будет матрица $C$ размером $m \times n$.

Стандартный алгоритм умножения матриц можно описать следующим образом. Инициализация --- создается матрица $C$ размером $m \times n$, и все её элементы инициализируются нулями. Это необходимо для того, чтобы избежать случайных значений в результирующей матрице. Вложенные циклы --- для вычисления каждого элемента матрицы $C$ используются три вложенных цикла, 
внешний цикл проходит по строкам матрицы $A$ (индекс $i$), 
cредний цикл проходит по столбцам матрицы $B$ (индекс $j$), 
внутренний цикл проходит по элементам строки матрицы $A$ и столбца матрицы $B$ (индекс $k$), вычисляя сумму произведений соответствующих элементов.

\begin{equation}
	\label{eq:std}
	C[i][j] = \sum_{k=0}^{k-1} A[i][k] \times B[k][j]
\end{equation}


\section{Алгоритм Винограда}

Алгоритм Винограда основан на использовании предварительных вычислений для уменьшения количества необходимых операций умножения. В отличие от стандартного алгоритма, который требует $O(m \cdot n \cdot p)$ операций умножения для умножения матриц $A$ размером $m×p$ и $B$ размером $p×n$, алгоритм Винограда снижает это количество до $O(m \cdot n+m+n)$, что делает его более эффективным для больших матриц.

Алгоритм начинает с вычисления промежуточных значений, которые позволяют сократить количество операций умножения. Для матриц $A$ и $B$ вычисляются два массива, для строк матрицы $A$: 
\begin{equation}
	\label{eq:grapeA}
	P[i] = \sum_{k=0}^{k-1} A[i][k] \cdot B[k][j]
\end{equation}	

Для столбцов матрицы $B$:
\begin{equation}
	\label{eq:grapeB}
	Q[j] = \sum_{k=0}^{k-1} A[i][k] \cdot B[k][j]
\end{equation}	

После вычисления промежуточных значений, алгоритм использует их для вычисления элементов результирующей матрицы $C$. Каждый элемент $C[i][j]$ вычисляется как:
\begin{equation}
	\label{eq:grapeRes}
	C[i][j] = P[i] + Q[j]
\end{equation}	

Затем выполняется сложения промежуточных значений, что позволяет избежать повторных вычислений и значительно ускоряет процесс умножения.


\section{Оптимизированный алгоритм Винограда}

Индивидуальный вариант: инкремент счётчика наиболее вложенного цикла на 2; использование инкремента (+=); введение декремента при вычислении вспомогательных массивов;