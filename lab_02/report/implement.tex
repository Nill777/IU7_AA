\chapter{Технологическая часть}

\section{Средства реализации}

В данной работе для реализации был выбран язык программирования $Python$. Требуется измерить затрачиваемое время и построить графики. Для построения графиков использовалась библиотека $mathplotlib$.

\section{Реализация алгоритмов}

В листингах~\ref{lst:mul},~\ref{lst:grape} и~\ref{lst:opt_grape} представлены реализации алгоритма стандартного умножения матриц, алгоритма Винограда, оптимизированного алгоритма Винограда.
\begin{center}
    \captionsetup{justification=raggedright,singlelinecheck=off}
    \begin{lstlisting}[label=lst:mul,caption=Стандартный алгоритм умножения матриц]
def mul_std(a: Matrix, b: Matrix):
	if (a.size != b.size):
		return ERROR_DIFFERENT_SIZE
	size = a.size
	res = Matrix(size, Gen.ZERO_MATRIX)
	for i in range(size):
		for j in range(size):
			for k in range(size):
				res.data[i][j] += a.data[i][k] * b.data[k][j]
	return res
\end{lstlisting}
\end{center}

\clearpage

\begin{center}
    \captionsetup{justification=raggedright,singlelinecheck=off}
    \begin{lstlisting}[label=lst:grape,caption=Алгоритм Винограда]
def mul_grape(a: Matrix, b: Matrix):
	if (a.size != b.size):
		return ERROR_DIFFERENT_SIZE
	
	size = a.size
	row_a = size
	col_a = size
	col_b = size
	tmp_row = [0] * row_a
	tmp_col = [0] * col_b
	
	for i in range(row_a):
		for j in range(0, col_a // 2):
			tmp_row[i] = tmp_row[i] + a.data[i][2 * j] * a.data[i][2 * j + 1]

	for i in range(col_b):
		for j in range(0, col_a // 2):
			tmp_col[i] = tmp_col[i] + b.data[2 * j][i] * b.data[2 * j + 1][i] 

	res = Matrix(size, Gen.ZERO_MATRIX)
	for i in range(row_a):
		for j in range(col_b):
			res.data[i][j] = -tmp_row[i] - tmp_col[i]   
			for k in range(0, col_a // 2):
				res.data[i][j] = res.data[i][j] + (a.data[i][2 * k + 1] + b.data[2 * k][j]) * (a.data[i][2 * k] + b.data[2 * k + 1][j])

	if (col_a % 2 == 1):
		for i in range(row_a):
			for j in range(col_b):
				res.data[i][j] = res.data[i][j] + a.data[i][col_a - 1] * b.data[col_a - 1][j]

	return res
\end{lstlisting}
\end{center}

\clearpage

\begin{center}
    \captionsetup{justification=raggedright,singlelinecheck=off}
    \begin{lstlisting}[label=lst:opt_grape,caption=Оптимизированный алгоритм Винограда]
def mul_grape_opt(a: Matrix, b: Matrix):
	if (a.size != b.size):
		return ERROR_DIFFERENT_SIZE
	
	size = a.size
	row_a = size
	col_a = size
	col_b = size
	tmp_row = [0] * row_a
	tmp_col = [0] * col_b
	
	for i in range(row_a):
		for j in range(0, col_a // 2):
			tmp_row[i] = tmp_row[i] + a.data[i][2 * j] * a.data[i][2 * j + 1]

	for i in range(col_b):
		for j in range(0, col_a // 2):
			tmp_col[i] = tmp_col[i] + b.data[2 * j][i] * b.data[2 * j + 1][i] 

	flag = col_a % 2
	res = Matrix(size, Gen.ZERO_MATRIX)
	for i in range(row_a):
		for j in range(col_b):
			res.data[i][j] -= tmp_row[i] + tmp_col[i]
			for k in range(1, col_a, 2):
				res.data[i][j] += (a.data[i][k - 1] + b.data[k][j]) * (a.data[i][k] + b.data[k - 1][j])
			if (flag):
				res.data[i][j] += a.data[i][col_a - 1] * b.data[col_a - 1][j]

	return res
\end{lstlisting}
\end{center}

\clearpage

\section{Функциональные тесты}

В таблице~\ref{tbl:functional_test} приведены тесты для функций программы. Тесты для всех функций пройдены успешно.

\begin{center}
    \captionsetup{justification=raggedright,singlelinecheck=off}
    \begin{longtable}[c]{|c|c|c|c|}
    \caption{Функциональные тесты\label{tbl:functional_test}} \\ \hline
		Матрица $A$ & Матрица $B$ & Ожидание & Результат \\ \hline
		$\begin{pmatrix}1 & 0 & 0 \\
						0 & 1 & 0 \\
						0 & 0 & 1 \end{pmatrix}$ & 
		$\begin{pmatrix}1 & 2 & 3 \\
						4 & 5 & 6 \\
						7 & 8 & 9 \end{pmatrix}$ & 
		$\begin{pmatrix}1 & 2 & 3 \\
						4 & 5 & 6 \\
						7 & 8 & 9 \end{pmatrix}$ &  
		$\begin{pmatrix}1 & 2 & 3 \\
						4 & 5 & 6 \\
						7 & 8 & 9 \end{pmatrix}$					
						\\ \hline

		$\begin{pmatrix}1 & 1 & 1 \\
						1 & 1 & 1 \\
						1 & 1 & 1 \end{pmatrix}$ & 
		$\begin{pmatrix}1 & 2 & 3 \\
						4 & 5 & 6 \\
						7 & 8 & 9 \end{pmatrix}$ &
		$\begin{pmatrix}12 & 15 & 18 \\
						12 & 15 & 18 \\
						12 & 15 & 18 \end{pmatrix}$ &
		$\begin{pmatrix}12 & 15 & 18 \\
						12 & 15 & 18 \\
						12 & 15 & 18 \end{pmatrix}$					
						\\ \hline

		$\begin{pmatrix}0 & 1 \\
						7 & 9 \end{pmatrix}$ & 
		$\begin{pmatrix}8 & 4 \\
						2 & 4 \end{pmatrix}$ & 
		$\begin{pmatrix}2 & 4 \\
						74 & 64 \end{pmatrix}$ & 
		$\begin{pmatrix}2 & 4 \\
						74 & 64 \end{pmatrix}$ 						
						\\ \hline

		$\begin{pmatrix}-2 & -1 & 0 \\
						3 & 2 & 1 \\
						-1 & 2 & -2 \end{pmatrix}$ & 
		$\begin{pmatrix}1 & -1 & 1 \\
						2 & 0 & 3 \\
						1 & 2 & 0 \end{pmatrix}$ & 
		$\begin{pmatrix}-4 & 2 & -5 \\
						8 & -1 & 9 \\
						1 & -3 & 5 \end{pmatrix}$ & 
		$\begin{pmatrix}-4 & 2 & -5 \\
						8 & -1 & 9 \\
						1 & -3 & 5 \end{pmatrix}$						
						\\ \hline

		$\begin{pmatrix}-4 & -1 & -2 \\
						-2 & -1 & -3 \\
						-2 & -3 & -1 \end{pmatrix}$ & 
		$\begin{pmatrix}3 & 1 & 2 \\
						1 & 1 & 2 \\
						1 & 4 & 3 \end{pmatrix}$ & 
		$\begin{pmatrix}-15 & -13 & -16 \\
						-10 & -15 & -15 \\
						-10 & -9 & -13 \end{pmatrix}$ & 
		$\begin{pmatrix}-15 & -13 & -16 \\
						-10 & -15 & -15 \\
						-10 & -9 & -13 \end{pmatrix}$ 						
						\\ \hline
	\end{longtable}
\end{center}

