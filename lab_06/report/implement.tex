\chapter{Технологическая часть}

\section{Средства реализации}

В данной работе для реализации был выбран язык программирования $Python$. Требуется измерить временные затраты и построить графики. Для построения графиков использовалась библиотека $mathplotlib$.

\section{Реализация алгоритмов}

В листингах~\ref{lst:brute_force}-~\ref{lst:ant} представлены реализации алгоритма полного перебора и муравьиного алгоритма для решения задачи коммивояжёра.
\begin{center}
    \captionsetup{justification=raggedright,singlelinecheck=off}
    \begin{lstlisting}[label=lst:brute_force,caption=Алгоритм полного перебора]
def brute_force_voyager(matrix, size):
	cities = list(range(size))
	permutations = it.permutations(cities)

	min_len = float("inf")
	best_route = None
	for perm in permutations:
		cur_route = list(perm)
		cur_len = 0

		for i in range(len(cur_route) - 1):
			cur_len += matrix[cur_route[i]][cur_route[i + 1]]

		if cur_len < min_len:
			min_len = cur_len
			best_route = cur_route

	return min_len, best_route
\end{lstlisting}
\end{center}

\clearpage

\begin{center}
    \captionsetup{justification=raggedright,singlelinecheck=off}
    \begin{lstlisting}[label=lst:ant,caption=Муравьиный алгоритм]
def daily_quota_Q(matrix, size):
	matrix = np.array(matrix)
	total_sum = np.sum(matrix) - np.sum(np.diag(matrix))
	count = size * (size - 1)
	return total_sum / count

def calc_len_route(matrix, route):
	length = sum(matrix[route[i], route[i + 1]] for i in range(len(route) - 1))
	return length

def update_pheromones(matrix, size, visited, pheromones, daily_quota, k_evaporation):
	for i in range(size):
		for j in range(size):
			delta_pheromones = 0
			for ant in range(size):
				length = calc_len_route(matrix, visited[ant])
				delta_pheromones += daily_quota / length

			pheromones[i][j] = pheromones[i][j] * (1 - k_evaporation) + delta_pheromones
			pheromones[i][j] = max(pheromones[i][j], MIN_PHEROMONE)

	return pheromones

def get_matrix_pheromones(size):
	return [[INIT_PHEROMONE] * size for _ in range(size)]

def get_matrix_visibility(matrix, size):
	visibility_matrix = []
	for i in range(size):
		row = []
		for j in range(size):
			if i != j:
				row.append(1.0 / matrix[i][j])
			else:
				row.append(0)
		visibility_matrix.append(row)

	return visibility_matrix

def get_matrix_visited(route, ants):
	visited = [[] for _ in range(ants)]
	for ant in range(ants):
		visited[ant].append(int(route[ant]))

	return visited

def choose_next_city(pheromones, ant, alpha, beta, visibility, visited, cities):
	probabilities = [0] * cities
	for city in range(cities):
		if city not in visited[ant]:
			ant_city = visited[ant][-1]
			probabilities[city] = (pheromones[ant_city][city] ** alpha) * (visibility[ant_city][city] ** beta)
		else:
			probabilities[city] = 0

	sum_probabilities = sum(probabilities)
	if sum_probabilities > 0:
		probabilities = [p / sum_probabilities for p in probabilities]
	else:
		return None

	posibility = random()
	cumulative_p = np.cumsum(probabilities)
	chosen_city = np.searchsorted(cumulative_p, posibility)
	
	return int(chosen_city)

def ant_alg(matrix, cities, alpha, beta, days, k_evaporation):
	best_route = []
	min_len = float("inf")
	daily_quota = daily_quota_Q(matrix, cities)
	pheromones = get_matrix_pheromones(cities)
	visibility = get_matrix_visibility(matrix, cities)
	ants = cities

	for day in range(days):
		route = np.arange(cities)
		visited = get_matrix_visited(route, ants)
		for ant in range(ants):
			while (len(visited[ant]) != ants):
				chosen_city = choose_next_city(pheromones, ant, alpha, beta, visibility, visited, cities)
				visited[ant].append(chosen_city)    # -1

			cur_len = calc_len_route(matrix, visited[ant]) 
			if (cur_len < min_len):
				min_len = cur_len
				best_route = visited[ant]

		pheromones = update_pheromones(matrix, cities, visited, pheromones, daily_quota, k_evaporation)

	return min_len, best_route
\end{lstlisting}
\end{center}


\section{Функциональные тесты}

В таблице~\ref{tbl:functional_test} приведены тесты для функций программы. Тесты для всех функций пройдены успешно.

\begin{center}
    \captionsetup{justification=raggedright,singlelinecheck=off}
    \begin{longtable}[c]{|c|c|c|c|c|}
    \caption{Функциональные тесты\label{tbl:functional_test}} \\ \hline
		Матрица смежности & Ожидание & Результат \\ \hline
		$\begin{pmatrix}0 & 1 & 7 \\
						1 & 0 & 9 \\
						7 & 9 & 0 \end{pmatrix}$ & [1, 0, 2] & [1, 0, 2] \\ \hline
		$\begin{pmatrix}0 & 4 & 4 & 2 \\
						4 & 0 & 1 & 6 \\
						4 & 1 & 0 & 1 \\
						2 & 6 & 1 & 0 \end{pmatrix}$ & [0, 3, 2, 1] & [0, 3, 2, 1] \\ \hline
		$\begin{pmatrix}0 & 6 & 3 & 8 & 10 \\
						6 & 0 & 5 & 7 & 8 \\
						3 & 5 & 0 & 2 & 3 \\
						8 & 7 & 2 & 0 & 2 \\
						10 & 8 & 3 & 2 & 0 \end{pmatrix}$ & [1, 0, 2, 3, 4] & [1, 0, 2, 3, 4] \\ \hline
		$\begin{pmatrix}0 & 2 & 2 & 8 & 2 & 3 \\
						2 & 0 & 7 & 1 & 1 & 9 \\
						2 & 7 & 0 & 9 & 3 & 5 \\
						8 & 1 & 9 & 0 & 6 & 9 \\
						2 & 1 & 3 & 6 & 0 & 6 \\
						3 & 9 & 5 & 9 & 6 & 0 \end{pmatrix}$ & [3, 1, 4, 2, 0, 5] & [3, 1, 4, 2, 0, 5] \\
		\hline
	\end{longtable}
\end{center}
