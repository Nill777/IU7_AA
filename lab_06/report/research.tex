\chapter{Исследовательская часть}

\section{Технические характеристики}

Технические характеристики устройства:

\begin{itemize}
	\item[---] операционная система Manjaro Linux x86\_64;
	\item[---] процессор Ryzen 5500U 6 ядер, тактовая частота 2.1 ГГц;
	\item[---] оперативная память 16 Гбайт.
\end{itemize}

При тестировании ноутбук был включён в сеть электропитания. Во время тестирования ноутбук был нагружен только системными приложениями окружения, а также системой тестирования.

\section{Сравнительный анализ временных затрат}

Замеры времени для каждой размерности матрицы проводились 20 раз. Результат замера --- среднее арифметическое время работы алгоритма, на вход подавались матрицы сгенерированные случайным образом.

\imgHeight{90mm}{time}{Временные затраты}

\section{Результаты проводимых исследований}

По полученным данным измерений временных затрат был сделан вывод о том, что алгоритм полного перебора имеет большую вычислительную сложность по сравнению с муравьиным алгоритмом, это хорошо видно при увеличении размерности матрицы. Алгоритм полного перебора является значительно менее эффективным по временным затратам, но его преимуществом является гарантированное нахождение глобального минимума.


\section{Класс данных 1}

Класс данных 1 представляет собой матрицу смежности размерностью 10, разброс длины путей $[1, 10]$.

$M_{1} = 
\begin{pmatrix}0 & 6 & 6 & 8 & 8 & 2 & 8 & 7 & 4 & 3 \\
			6 & 0 & 3 & 4 & 7 & 4 & 6 & 9 & 2 & 1 \\
			6 & 3 & 0 & 5 & 9 & 2 & 8 & 4 & 4 & 6 \\
			8 & 4 & 5 & 0 & 4 & 2 & 2 & 1 & 4 & 7 \\
			8 & 7 & 9 & 4 & 0 & 8 & 8 & 4 & 6 & 3 \\
			2 & 4 & 2 & 2 & 8 & 0 & 9 & 5 & 1 & 1 \\
			8 & 6 & 8 & 2 & 8 & 9 & 0 & 3 & 9 & 9 \\
			7 & 9 & 4 & 1 & 4 & 5 & 3 & 0 & 9 & 8 \\
			4 & 2 & 4 & 4 & 6 & 1 & 9 & 9 & 0 & 7 \\
			3 & 1 & 6 & 7 & 3 & 1 & 9 & 8 & 7 & 0 \end{pmatrix}$

Для данного класса данных при параметризации для каждого набора параметров проводилось 15 измерений результаты приведены в приложении А. Наилучшим набором является набор коэффициентов под номером 10.


\section{Класс данных 2}

Класс данных 2 представляет собой матрицу смежности размерностью 10, разброс длины путей $[1, 1000]$.

$M_{2} = 
\begin{pmatrix}0 & 329 & 358 & 948 & 241 & 778 & 103 & 718 & 204 & 114 \\
			329 & 0 & 981 & 614 & 80 & 313 & 19 & 87 & 410 & 539 \\
			358 & 981 & 0 & 277 & 834 & 427 & 100 & 265 & 535 & 818 \\
			948 & 614 & 277 & 0 & 802 & 132 & 34 & 565 & 375 & 91 \\
			241 & 80 & 834 & 802 & 0 & 150 & 224 & 121 & 881 & 879 \\
			778 & 313 & 427 & 132 & 150 & 0 & 240 & 466 & 587 & 154 \\
			103 & 19 & 100 & 34 & 224 & 240 & 0 & 990 & 98 & 1000 \\
			718 & 87 & 265 & 565 & 121 & 466 & 990 & 0 & 833 & 384 \\
			204 & 410 & 535 & 375 & 881 & 587 9& 8 & 833 & 0 & 845 \\
			114 & 539 & 818 & 91 & 879 & 154 & 1000 & 384 & 845 & 0 \end{pmatrix}$

Для данного класса данных при параметризации для каждого набора параметров проводилось 15 измерений результаты приведены в приложении А. Наилучшим набором является набор коэффициентов под номером 5.


\section{Класс данных 3}

Класс данных 3 представляет собой матрицу смежности размерностью 10, разброс длины путей $[1, 10000]$.

$M_{3} = 
\begin{pmatrix}0 & 4886 & 3635 & 2763 & 5266 & 3840 & 6731 & 4223 & 2986 & 8571 \\
			4886 & 0 & 3555 & 8981 & 7726 & 4041 & 7116 & 2701 & 2271 & 4951 \\
			3635 & 3555 & 0 & 169 & 5813 & 7148 & 3570 & 2560 & 6898 & 8612 \\
			2763 & 8981 & 169 & 0 & 472 & 4641 & 2948 & 5613 & 7216 & 4996 \\
			5266 & 7726 & 5813 & 472 & 0 & 2893 & 8984 & 6664 & 3508 & 6807 \\
			3840 & 4041 & 7148 & 4641 & 2893 & 0 & 3981 & 5081 & 1726 & 2477 \\
			6731 & 7116 & 3570 & 2948 & 8984 & 3981 & 0 & 1746 & 7391 & 1641 \\
			4223 & 2701 & 2560 & 5613 & 6664 & 5081 & 1746 & 0 & 4066 & 7988 \\
			2986 & 2271 & 6898 & 7216 & 3508 & 1726 & 7391 & 4066 & 0 & 1177 \\
			8571 & 4951 & 8612 & 4996 & 6807 & 2477 & 1641 & 7988 & 1177 & 0 \end{pmatrix}$
			
Для данного класса данных при параметризации для каждого набора параметров проводилось 15 измерений результаты приведены в приложении А. Наилучшим набором является набор коэффициентов под номером 15.

\clearpage
