\chapter{Аналитическая часть}
\section{Задача коммивояжёра}

Задача коммивояжёра формулируется следующим образом: необходимо найти такой кратчайший путь по заданным $n$ городам, чтобы каждый город посещался только один раз. Проблема моделируется при помощи взвешенного графа, вершины которого представляют города, а веса рёбер определяют расстояния~\cite{Levitin2006}.


\section{Алгоритм полного перебора}

Алгоритм полного перебора~\cite{Boroznov2009} осуществляет поиск в пространстве $N!$ решений посредством перебора всех вариантов маршрутов. Преимуществом данного алгоритма заключается в том, что он гарантированно находит лучшее решение(глобальный минимум). Недостатком алгоритма полного перебора является его временная сложность --- $O(N!)$, поэтому данный алгоритм целесообразно использовать, когда $N$ является малым.


\section{Муравьиный алгоритм}

Муравьиный алгоритм~\cite{Shtovba2003} представляет собой вероятностную жадную эвристику, где вероятности устанавливаются, исходя из информации о качестве решения, полученной из предыдущих решений. Идея муравьиного алгоритма --- моделирование поведения муравьёв, связанного с их способностью быстро находить кратчайший путь от муравейника к источнику пищи и адаптироваться к изменяющимся условиям, находя новый кратчайший путь. При своём движении муравей помечает путь феромоном, и эта информация используется другими муравьями для выбора пути.

У муравья три компетенции:
\begin{itemize}
	\item[---] зрение --- муравей может определить привлекательность ребра;
	\item[---] память --- запоминает каждый посещённый в текущий день город в кортеж;
	\item[---] обоняние --- муравей чует концентрацию феромона на ребре.
\end{itemize}

Видимость --- величина, обратная расстоянию: $\eta_{ij} = 1/D_{ij}$, где $D_{ij}$ --- расстояние между городами i и j.

Вероятностно-пропорциональное правило, определяющее вероятность перехода муравья $k$ из текущей вершины $i$ в вершину $j$ на $t$ итерации рассчитывается по формуле~\eqref{p}:

\begin{equation}
    \label{p}
    P_{kij}(t) = 
    \begin{cases}
        \frac{\eta_{ij}^\alpha (\tau_{ij}(t))^\beta}{\sum_{q=1}^N \eta_{ij}^\alpha (\tau_{iq}(t))^\beta}, \text{если вершина $j$ ещё не посещена муравьём $k$,} \\
        0, \text{иначе,}
    \end{cases}
\end{equation}

где $\tau_{ij}$ --- количество феромонов на ребре $(i, j)$, $\alpha$ --- коэффициент жадности алгоритма, $\beta$ --- коэффициент стадности алгоритма.

После завершения движения всех муравьев осуществляется пересчёт уровня феромона для каждого ребра по следующей формуле~\eqref{pheromone}:

\begin{equation}
    \label{pheromone}
    \tau_{ij}(t+1) = (1-p)\tau_{ij}(t) + \Delta\tau_{ij}(t),
\end{equation}

где $p \in (0, 1)$ — коэффициент испарения феромона, $\Delta\tau_{ij}(t)$ вычисляется по формуле:

\begin{equation}
    \label{delta_tau}
    \Delta\tau_{ij} = \sum_{k=1}^N \Delta\tau_{ijk}(t),
\end{equation}

\begin{equation}
    \label{delta_tau_k}
    \Delta\tau_{ijk}(t) = 
    \begin{cases}
        0, \text{если по ребру $i-j$ муравей $k$ в день $t$ не ходил,}\\
        \frac{Q}{L_{k}}(t), \text{иначе}
    \end{cases}
\end{equation}

где $Q$ --- дневная квота феромона муравья, величина соизмеримая длине лучшего маршрута, а $L_k$ --- длина маршрута муравья $k$.

