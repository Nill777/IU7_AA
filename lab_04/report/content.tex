% Содержимое отчета по курсу Анализ алгоритмов

\aaunnumberedsection{ВВЕДЕНИЕ}{sec:intro}

Параллельные вычисления – это использование нескольких или многих вычислительных устройств для одновременного выполнения разных частей одной программы~\cite{gafarov_paral}.

Цель работы --- получение навыка организации параллельных вычислений на основе нативных потоков. 

Задачи работы: 
\begin{itemize}
	\item[---] анализ структуры сайта;
	\item[---] разработка алгоритма обработки данных;
	\item[---] разработка программного обеспечения, скачивающего html страницы в однопоточном и многопоточном режиме;
	\item[---] исследование времени выгрузки html страниц от числа обрабатывающих потоков;
\end{itemize}


\aasection{Входные и выходные данные}{sec:input-output}

Входным данным является ссылка на сайт https://www.gotovim.ru, который содержит множество рецептов различных блюд. Выходными данными являются html файлы --- скачанные страницы рецептов.


\aasection{Преобразование входных данных в выходные}{sec:algorithm}

Программа отправляет запрос по исходной ссылке, разбирает полученную страницу и извлекает ссылки на категории блюд. Отправляются запросы по ссылкам на категории, из полученных страниц извлекаются ссылки на конкретные рецепты. Далее загружаются html страницы рецептов в директорию /data.

\includelisting
{parse.cpp} % Имя файла с расширением (файл должен быть расположен в директории inc/lst/)
{Функции извлечения ссылок на рецепты} % Подпись листинга

\aasection{Примеры работы программы}{sec:demo}
На рисунке~\ref{img:exemple} представлен пример работы программы.

\FloatBarrier
\includeimage
{exemple} % Имя файла без расширения (файл должен быть расположен в директории inc/img/)
{f} % Обтекание (без обтекания)
{h} % Положение рисунка (см. figure из пакета float)
{\textwidth} % Ширина рисунка
{Пример работы программы} % Подпись рисунка
\FloatBarrier


\aasection{Тестирование}{sec:tests}

Тестирование программы проводилось загрузкой 50 страниц по ссылкам на различные рецепты. Проверялось количество удачных загрузок и соответствие выходных файлов исходным html страницам. 

Как для однопоточной реализации, так и для многопоточной тестировалось соответствие исходного перечня страниц и загруженных файлов, если все страницы были успешно загружены, тест считался успешным, в обоих реализация файлы были получены идентичные. Тестирование было пройдено успешно.


\aasection{Описание исследования}{sec:study}

Технические характеристики устройства:

\begin{itemize}
	\item[---] операционная система Manjaro Linux x86\_64;
	\item[---] процессор Ryzen 5500U 6 ядер, тактовая частота 2.1 ГГц;
	\item[---] оперативная память 16 Гбайт.
\end{itemize}

В ходе исследования был получен график зависимости времени выполнения от количества задействованных потоков от 0 (вычисление в основном потоке), до $4\cdot k$, где $k = 12$ --- количество логических ядер.

Замер проводился 10 раз для каждого числа потоков, загружалось 50 ссылок и вычислялось среднее арифметическое время работы. На рисунке~\ref{img:graph} представлен график полученной зависимости. Время указано в миллисекундах, количество потоков в штуках.

\FloatBarrier
\includeimage
{graph} % Имя файла без расширения (файл должен быть расположен в директории inc/img/)
{f} % Обтекание (без обтекания)
{h} % Положение рисунка (см. figure из пакета float)
{\textwidth} % Ширина рисунка
{График зависимости время выполнения от количества потоков} % Подпись рисунка
\FloatBarrier

По результатам проведенного исследования сделан вывод о том, что время загрузки html страниц зависит не только от количества потоков, которые выполняют задачу, но и от скорости работы/загруженности сервера.

\aaunnumberedsection{ЗАКЛЮЧЕНИЕ}{sec:outro}

Получены навыки организации параллельных вычислений на основе нативных потоков. Цель работы достигнута. Решены все поставленные задачи: 
\begin{itemize}
	\item[---] анализ структуры сайта;
	\item[---] разработка алгоритма обработки данных;
	\item[---] разработка программного обеспечения, скачивающего html страницы в однопоточном и многопоточном режиме;
	\item[---] исследование времени выгрузки html страниц от числа обрабатывающих потоков;
\end{itemize}
